\RequirePackage[l2tabu,orthodox]{nag}            %% Warn about obsolete commands and packages
\RequirePackage{amsmath,amsfonts,amssymb,amsthm} %% Math
\RequirePackage{ifpdf,ifxetex,ifluatex}          %% Detect XeTeX and LuaTeX
\RequirePackage{xspace}
\RequirePackage{graphicx}
\RequirePackage{comment}
\RequirePackage{url}
\RequirePackage{relsize}
\RequirePackage{booktabs}
\RequirePackage{tabularx}
\RequirePackage[normalem]{ulem}
\ifluatex%
\else%
  \RequirePackage[all]{xy}
\fi%
\RequirePackage{etoolbox}
\RequirePackage{csquotes}
\RequirePackage[export]{adjustbox}

\RequirePackage{silence}
\WarningsOff[microtype]
\WarningFilter{microtype}{Unknown slot}

% https://tex.stackexchange.com/questions/64459/overfull-vbox-warning-disable
\vfuzz=30pt
\hfuzz=30pt


%%%
%%% Code Listings
%%%

\RequirePackage{listings}
\lstdefinelanguage{Sage}[]{Python}{morekeywords={True,False,sage,cdef,cpdef,ctypedef,self},sensitive=true}

\lstset{frame=none,
  showtabs=False,
  showspaces=False,
  showstringspaces=False,
  commentstyle={\color{gray}},
  keywordstyle={\color{mLightBrown}\textbf},
  stringstyle ={\color{mDarkBrown}},
  frame=single,
  basicstyle=\tt\scriptsize\relax,
  backgroundcolor=\color{gray!190!black},
  inputencoding=utf8,
  literate={…}{{\ldots}}1,
  belowskip=0.0em,
}

\makeatletter
\patchcmd{\@verbatim}
  {\verbatim@font}
  {\verbatim@font\scriptsize}
  {}{}
\makeatother


%%%
%%% Pseudocode
%%%

\let\nl\undefine
\let\procedure\relax
\let\endprocedure\relax
\usepackage{algorithm2e}

%%%
%%% Tikz
%%%

\RequirePackage{tikz,pgfplots}
\pgfplotsset{compat=newest}

\usetikzlibrary{calc}
\usetikzlibrary{arrows}
\usetikzlibrary{automata}
\usetikzlibrary{positioning}
\usetikzlibrary{decorations.pathmorphing}
\usetikzlibrary{backgrounds}
\usetikzlibrary{fit,}
\usetikzlibrary{shapes.symbols}
\usetikzlibrary{chains}
\usetikzlibrary{shapes.geometric}
\usetikzlibrary{shapes.arrows}
\usetikzlibrary{graphs}

%% Cache but disable by default

\usetikzlibrary{external}
\tikzset{external/export=false}

\definecolor{DarkPurple}{HTML}{332288}
\definecolor{DarkBlue}{HTML}{6699CC}
\definecolor{LightBlue}{HTML}{88CCEE}
\definecolor{DarkGreen}{HTML}{117733}
\definecolor{DarkRed}{HTML}{661100}
\definecolor{LightRed}{HTML}{CC6677}
\definecolor{LightPink}{HTML}{AA4466}
\definecolor{DarkPink}{HTML}{882255}
\definecolor{LightPurple}{HTML}{AA4499}
\definecolor{DarkBrown}{HTML}{604c38}
\definecolor{DarkTeal}{HTML}{23373b}
\definecolor{LightBrown}{HTML}{EB811B}
\definecolor{LightGreen}{HTML}{14B03D}
\definecolor{DarkOrange}{HTML}{FFDD00}

\pgfplotsset{width=1.0\textwidth,
  height=0.6\textwidth,
  cycle list={%
    solid,LightGreen,thick\\%
    dotted,LightRed,very thick\\%
    dashed,DarkBlue,thick\\%
    dashdotted,DarkPink,thick\\%
    dashdotdotted,LightGreen,thick\\%
    loosely dotted,very thick\\%
    loosely dashed,DarkBlue,very thick\\%
    loosely dashdotted,DarkPink,very thick\\%
    \\%
    DarkBrown,thick\\%
  },
  legend pos=north west,
  legend cell align={left}}

\pgfplotsset{select coords between index/.style 2 args={
    x filter/.code={
        \ifnum\coordindex<#1\def\pgfmathresult{}\fi
        \ifnum\coordindex>#2\def\pgfmathresult{}\fi
    }
}}

\setlength{\marginparwidth}{2cm}
\pgfplotsset{compat=1.18}

%%%
%%% SVG (Inkscape)
%%%

\ifpdf% 
\providecommand{\executeiffilenewer}[3]{%
  \ifnum\pdfstrcmp{\pdffilemoddate{#1}}%
    {\pdffilemoddate{#2}}>0%
    {\immediate\write18{#3}}
  \fi%
}
\else%
\providecommand{\executeiffilenewer}[3]{%
  {\immediate\write18{#3}} % hack
}
\fi%

\providecommand{\includesvg}[2][1.0\textwidth]{%
 \executeiffilenewer{#1.svg}{#1.pdf}%
 {inkscape -z -D --file=#2.svg --export-pdf=#2.pdf --export-latex --export-area-page}%
 \def\svgwidth{#1} 
 \input{#2.pdf_tex}%
} 

%%%
%%% Attachments
%%%

\RequirePackage{embedfile}


%%%
%%% Metropolis Theme
%%%

\usetheme{metropolis}
\metroset{color/block=fill}
\metroset{numbering=none}
\metroset{outer/progressbar=foot}
\metroset{titleformat=smallcaps}

\setbeamercolor{description item}{fg=mLightBrown}
\setbeamerfont{footnote}{size=\scriptsize}
\setbeamercolor{example text}{fg=mDarkBrown}
\setbeamercolor{block title alerted}{fg=white, bg=mDarkBrown}
\setbeamerfont{alerted text}{series=\ifmmode\boldmath\else\bfseries\fi}

\renewcommand*{\UrlFont}{\ttfamily\relax}

%%%
%%% UTF-8 & Fonts
%%% 

\RequirePackage{unicodesymbols} % after metropolis which loads fontspec

\ifboolexpr{bool{xetex} or bool{luatex}}{%
\setmonofont[BoldFont={Cousine Bold},
             ItalicFont={Cousine Italic},
             BoldItalicFont={Cousine Bold Italic},
             Scale=0.9]{Cousine}             
}{%
}

%%%
%%% BibLaTeX
%%%

\RequirePackage[backend=bibtex,
            style=alphabetic,
            maxnames=8,maxbibnames=8,maxalphanames=8,
            citestyle=alphabetic]{biblatex}

\bibliography{local.bib,abbrev3.bib,crypto_crossref.bib,rfc.bib,jacm.bib,dcc.bib}

\setbeamertemplate{bibliography item}[text]
% https://tex.stackexchange.com/questions/683533/beamer-theme-metropolis-does-not-allow-different-font-size-for-fullcite
\setbeamerfont{bibliography entry title}{size=}
\setbeamerfont{bibliography entry author}{size=}
\setbeamerfont{bibliography entry location}{size=}
\setbeamerfont{bibliography entry note}{size=}

\DeclareFieldFormat{title}{\alert{#1}}
\DeclareFieldFormat[book]{title}{\alert{#1}}
\DeclareFieldFormat[thesis]{title}{\alert{#1}}
\DeclareFieldFormat[inproceedings]{title}{\alert{#1}}
\DeclareFieldFormat[incollection]{title}{\alert{#1}}
\DeclareFieldFormat[article]{title}{\alert{#1}}
\DeclareFieldFormat[misc]{title}{\alert{#1}}

%%% 
%%% Microtype
%%%

\IfFileExists{upquote.sty}{\RequirePackage{upquote}}{}
\IfFileExists{microtype.sty}{\RequirePackage{microtype}}{}
\IfFileExists{microtype.sty}{\PassOptionsToPackage{verbose=silent}{microtype}}{}

\setlength{\parindent}{0pt}                   %%
\setlength{\parskip}{6pt plus 2pt minus 1pt}  %%
\setlength{\emergencystretch}{3em}            %% prevent overfull lines
\setcounter{secnumdepth}{0}                   %%

%%%
%%% Maths
%%%

\DeclareMathOperator{\Vol}{Vol}
\DeclareMathOperator{\GH}{GH}
\renewcommand{\vec}[1]{\ensuremath{\mathbf{#1}}\xspace}
\newcommand{\norm}[1]{\left\lVert#1\right\rVert}
\providecommand{\mat}[1]{\ensuremath{\vec{#1}}\xspace}
\providecommand{\ring}[0]{\ensuremath{\mathcal{R}}\xspace}

